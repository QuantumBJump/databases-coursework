\documentclass{report}
\usepackage{listings}
\usepackage{courier}
\author{Quinn Stevens}
\title{NHS Prescription Information from January and February 2016}

\lstdefinestyle{mystyle}{
	breaklines=true,
	keepspaces=true,
	basicstyle=\footnotesize\ttfamily
}
\lstset{style=mystyle}

\begin{document}
\maketitle
\section*{Executive Summary}
This report details the steps taken to analyse a dataset of NHS prescriptions in January and February 2016.
\tableofcontents
\chapter{Introduction}
Lorem ipsum dolor sit amet etc etc etc
\chapter{Setting up the Database}

In order to set up the database I downloaded the XAMPP database utility \cite{xampp} so I could use its shell program to run MySQL on my computer. I set up XAMPP and my own user account (in order for some of the commands to work properly I needed to use the \texttt{mysqladmin.exe -u root password [password]} command to set a root password and then \texttt{mysql -u root -p} to log in using that password.

Once logged in to MySQL on the XAMPP shell I created the database using the \texttt{CREATE DATABASE prescriptionsdb;} command, then switched to that database with \texttt{\textbackslash u prescriptionsdb}.
\chapter{Importing the Data}
I downloaded the csv files to import from the NHS website \cite{nhsprescriptiondata, nhspatientdata} and looked through them manually to wrk out how to set up the tables.

\section{Setting up the tables}
I decided to combine the data from January and February together, which left me with four tables to set up: \texttt{address} to keep track of practice names and addresses, \texttt{prescriptions} for data on the actual prescriptions, \texttt{chems} for details of the chemicals in use, and \texttt{patients} for the number of patients registered at each practice.

\subsection{Chems table}
For \texttt{chems}, the csv files were not set up in a way which allowed easy import, as they had three columns, but the third column wasn't named properly and wasn't populated. The three columns I made in my version of the table were \texttt{chem\_sub}, \texttt{name} and \texttt{period}, to keep track of the chemical's unique ID, name, and the month the entry was from respectively. I set this up with the following command:

\begin{lstlisting}[language=SQL, caption=Creating the chems table]
CREATE TABLE chems(Chem_Sub VARCHAR(255), Name VARCHAR(255), Period INT);
\end{lstlisting}

\subsection{Patients table}
For the \texttt{patients} table, the creation command was much longer, as there were many columns that weren't actually needed to answer the question. The command I used was:
\newpage
\begin{lstlisting}[language=SQL, caption=Creating the patients table]
CREATE TABLE patients(Practice_ID VARCHAR(255), Postcode VARCHAR(255), ONS_CCG_Code VARCHAR(255), CCG_Code VARCHAR(255), ONS_Region_Code VARCHAR(255), NHSE_Region_Code VARCHAR(255), ONS_Comm_Rgn_Code VARCHAR(255), NHSE_Comm_Rgn_Code VARCHAR(255), Total_All INT, Total_Male INT, Total_Female INT, Male_0_4 INT, Male_5_9 INT, Male_10_14 INT, Male_15_19 INT, Male_20_24 INT, Male_25_29 INT, Male_30_34 INT, Male_35_39 INT, Male_40_44 INT, Male_45_49 INT, Male_50_54 INT, Male_55_59 INT, Male_60_64 INT, Male_65_69 INT, Male_70_74 INT, Male_75_79 INT, Male_80_84 INT, Male_85_89 INT, Male_90_94 INT, Male_95_Up INT, Female_0_4 INT, Female_5_9 INT, Female_10_14 INT, Female_15_19 INT, Female_20_24 INT, Female_25_29 INT, Female_30_34 INT, Female_35_39 INT, Female_40_44 INT, Female_45_49 INT, Female_50_54 INT, Female_55_59 INT, Female_60_64 INT, Female_65_69 INT, Female_70_74 INT, Female_75_79 INT, Female_80_84 INT, Female_85_89 INT, Female_90_94 INT, Female_95_Up INT);
\end{lstlisting}
This created a very bloated table, which I later trimmed down to the essentials once the data had been imported.

\subsection{Prescriptions table}
The \texttt{prescriptions} table was relatively simple to create, as all the necessary columns were marked clearly in the csv file. I used this command to create the table:

\begin{lstlisting}[language=SQL, caption=Creating the prescriptions table]
CREATE TABLE prescriptions(SHA VARCHAR(255), PCT VARCHAR(255), Practice_ID VARCHAR(255), BNF_Code VARCHAR(255), BNF_Name VARCHAR(255), Items INT, NIC DOUBLE, ACT_Cost DOUBLE, Quantity INT, Period INT);
\end{lstlisting}

\subsection{Address table}
Finally, for the \texttt{address} table I needed to use my own initiative to come up with the column names, as they weren't labelled in the csv file. Most of them were obvious, but specifically I chose to label the columns containing the practice address as \texttt{Address\_1} through \texttt{Address\_4} as it wasn't consistent which part of the address was stored in each column, so naming them \texttt{Street}, \texttt{Town} etc would not have worked. In the end I used this command:

\begin{lstlisting}[language=SQL, caption=Creating the address table]
CREATE TABLE address(Period VARCHAR(255), Practice_ID VARCHAR(255), Practice_Name VARCHAR(255), Address_1 VARCHAR(255), Address_2 VARCHAR(255), Address_3 VARCHAR(255), Address_4 VARCHAR(255), Postcode VARCHAR(255));
\end{lstlisting}

\section{Populating and cleaning up the tables}
Populating the \texttt{prescriptions}, \texttt{address} and \texttt{patients} tables was easy; I simply used the following command for each (without using \lstinline{IGNORE 1 LINES;} for \texttt{address}, as that csv did not have column labels):

\begin{lstlisting}[language=SQL, caption=Loading data into tables]
LOAD DATA INFILE 'filename.csv'
INTO TABLE tablename
FIELDS TERMINATED BY ','
LINES TERMINATED BY '\n'
IGNORE 1 LINES;
\end{lstlisting}

Due to how bloated the \texttt{patients} table was with unnecessary data, I then used this command to trim it down:

\begin{lstlisting}[language=SQL, caption=Trimming down the patients table]
ALTER TABLE patients DROP COLUMN ONS_CCG_Code, DROP COLUMN CCG_Code, DROP COLUMN ONS_Region_Code, DROP COLUMN NHSE_Region_Code, DROP COLUMN ONS_Comm_Rgn_Code, DROP COLUMN NHSE_Comm_Rgn_Code, DROP COLUMN Total_Male, DROP COLUMN Total_Female, DROP COLUMN Male_0_4, DROP COLUMN Male_5_9, DROP COLUMN Male_10_14, DROP COLUMN Male_15_19, DROP COLUMN Male_20_24, DROP COLUMN Male_25_29, DROP COLUMN Male_30_34, DROP COLUMN Male_35_39, DROP COLUMN Male_40_44, DROP COLUMN Male_45_49, DROP COLUMN Male_50_54, DROP COLUMN Male_55_59, DROP COLUMN Male_60_64, DROP COLUMN Male_65_69, DROP COLUMN Male_70_74, DROP COLUMN Male_75_79, DROP COLUMN Male_80_84, DROP COLUMN Male_85_89, DROP COLUMN Male_90_94, DROP COLUMN Male_95_Up, DROP COLUMN Female_0_4, DROP COLUMN Female_5_9, DROP COLUMN Female_10_14, DROP COLUMN Female_15_19, DROP COLUMN Female_20_24, DROP COLUMN Female_25_29, DROP COLUMN Female_30_34, DROP COLUMN Female_35_39, DROP COLUMN Female_40_44, DROP COLUMN Female_45_49, DROP COLUMN Female_50_54, DROP COLUMN Female_55_59, DROP COLUMN Female_60_64, DROP COLUMN Female_65_69, DROP COLUMN Female_70_74, DROP COLUMN Female_75_79, DROP COLUMN Female_80_84, DROP COLUMN Female_85_89, DROP COLUMN Female_90_94, DROP COLUMN Female_95_Up;
\end{lstlisting}
This made the table a lot easier to manage.

Finally, the \texttt{chems} table needed a little more manual manipulation. I imported the January file with \lstinline{LOAD DATA INFILE '201601CHEM SUBS.csv' FIELDS TERMINATED BY ',' LINES TERMINATED BY '\n' IGNORE 1 LINES;}, then used the code in listing 3.7 to change the period to be correct. I then did the same with the february file, replacing 201601 with 201602.

\begin{lstlisting}[language=SQL, caption=Fixing period field in chems table]
UPDATE chems SET Period=201601 WHERE Period=0;
\end{lstlisting}

\chapter{Making the Queries}
\section{Question 1: How many practices and registered patients are there in the N17 postcode area?}
\paragraph{Interpretation}
I interpreted this question to be asking for the number of practices with a postcode beginning with 'N17', and the total number of patients registered at those practices.
\paragraph{Queries}
\begin{lstlisting}[language=SQL, caption=Question 1 query]
SELECT COUNT(Postcode) AS practices_in_area, SUM(Total_all) AS patients_in_area FROM patients WHERE POSTCODE LIKE 'N17%';
\end{lstlisting}
\paragraph{Results}
\begin{center}
\begin{tabular}{ | c | c | }
\hline
practices\textunderscore in\textunderscore area & patients\textunderscore in\textunderscore area \\
\hline
7 & 52248 \\
\hline
\end{tabular}
\end{center}
\paragraph{Conclusion}
There are 7 practices in the N17 postal area, with a total of  52,248 patients registered between them.

\section{Question 2: Which practice prescribed the most beta-blockers per registered patient in total over the two month period?}
\paragraph{Interpretation}
I interpreted this question to be asking for which practice prescribed the largest \textit{quantity} of beta-blockers per patient registered at that practice. The beta blockers I searched for were listed on the NHS website \cite{nhsbetablockers}:
\begin{itemize}
\item atenolol
\item bisoprolol
\item carvedilol
\item metroprolol
\item nebivolol
\item propranolol
\item tenormin
\item cardicor
\item emcor
\item betaloc
\item lopresor
\item nebilet
\item inderal
\end{itemize}
\paragraph{Queries}
\begin{lstlisting}[language=SQL, caption=Question 2 query]
SELECT 
    (SELECT address.practice_name FROM address WHERE address.Practice_ID = prescriptions.PRACTICE_ID group by practice_id) AS Practice_name, 
    prescriptions.practice_id AS practice_id,
	(SELECT Total_All FROM patients WHERE patients.Practice_ID = prescriptions.Practice_ID) AS registered_patients,
    (SUM(quantity) / (SELECT Total_All FROM patients WHERE patients.Practice_id = prescriptions.practice_id)) AS betablockers_per_patient 
    FROM prescriptions 
    WHERE bnf_name LIKE '%atenolol%' 
        OR bnf_name LIKE '%bisoprolol%' 
        OR bnf_name LIKE '%carvedilol%' 
        OR bnf_name LIKE '%metroprolol%' 
        OR bnf_name LIKE '%nebivolol%' 
        OR bnf_name LIKE '%propranolol%' 
        OR bnf_name LIKE '%tenormin%' 
        OR bnf_name LIKE '%cardicor%' 
        OR bnf_name LIKE '%emcor%' 
        OR bnf_name LIKE '%betaloc%' 
        OR bnf_name LIKE '%lopresor%' 
        OR bnf_name LIKE '%nebilet%' 
        OR bnf_name LIKE '%inderal%' 
    GROUP BY Practice_ID 
    ORDER BY betablockers_per_patient DESC 
    LIMIT 1;
\end{lstlisting}
\paragraph{Results}
\begin{center}
\begin{tabular}{ | c | c | c | c | }
\hline
Practice Name & Practice ID & Registered Patients & Betablockers per Patient \\
\hline
Burrswood Nursing Home & G82651 & 1 & 161.0000 \\
\hline
\end{tabular}
\end{center}
\paragraph{Conclusion}
The practice which prescribed the most beta-blockers per patient was burrswood nursing home. This seems to be because there is only one registered patient at burrswood nursing home.

\section{Question 3: Which was the most prescribed medication across all practices?}
\paragraph{Interpretation}
I interpreted this question to be asking for the greatest number of separate prescriptions (items) given for an individual chemical (\texttt{chem\_sub} in the \texttt{chems} table, or the first nine digits of \texttt{bnf\_code} in the \texttt{prescriptions} table)
\paragraph{Queries}
\begin{lstlisting}[language=SQL, caption=Question 3 query]
SELECT 
    (SELECT name 
        FROM chems 
        WHERE SUBSTR(prescriptions.bnf_code, 1, 9)=chems.chem_sub 
        GROUP BY SUBSTR(prescriptions.bnf_code, 1, 9)) 
    as chem_name, 
    COUNT(items) as items_prescribed 
    FROM prescriptions 
    GROUP BY SUBSTR(prescriptions.bnf_code, 1, 9) 
    ORDER BY items_prescribed DESC 
    LIMIT 1;
\end{lstlisting}
\paragraph{Results}
\begin{center}
\begin{tabular}{ | c | c | }
\hline
Chem Name & Items Prescribed \\
\hline
Colecalciferol & 280495 \\
\hline
\end{tabular}
\end{center}
\paragraph{Conclusion}
The most prescribed drug across all practices in January \& February 2016 was Colecalciferol - Colecalciferol is a form of Vitamin D, which implies that England has terrible weather, meaning that many people are unable to get sufficient sunlight and thus suffer Vitamin D deficiencies.

\section{Question 4: Which practice spent the most and least per patient?}
\paragraph{Interpretation}
I interpreted this question to be asking for the practices with the highest and lowest values of \texttt{SUM(act\_cost) / total\_all}, or: actual cost of all prescriptions divided by number of patients registered at that practice.
\paragraph{Queries}
This query returns many results with \texttt{total\_all = NULL}, and I couldn't find a way to exlude these results from the query, so I used the first result and the last non-NULL result as my answers to this question.
\begin{lstlisting}[language=SQL, caption=Question 4 query]
SELECT
    (SELECT practice_name 
        FROM address 
        WHERE address.practice_id = prescriptions.practice_id 
        GROUP BY practice_id) 
    AS Name,
    SUM(act_cost) AS total_cost,
    (SELECT Total_All 
        FROM patients 
        WHERE patients.practice_ID=prescriptions.practice_ID 
        GROUP BY practice_ID) 
    AS patients_registered,
    SUM(act_cost) / (SELECT Total_all 
        FROM patients 
        WHERE patients.practice_ID=prescriptions.practice_id 
        GROUP BY practice_id) 
    AS price_per_patient 
    FROM prescriptions 
    GROUP BY practice_ID 
    ORDER BY price_per_patient DESC;
\end{lstlisting}
\paragraph{Results}
\begin{center}
\begin{tabular}{ | c | c | c | c | }
\hline
Name & Total Cost & Patients Registered & Price per Patient \\
\hline
Burrswood Nursing Home & 7609.05 & 1 & 7609.05 \\
\hline
\ldots & \ldots & \ldots & \ldots \\
\hline
School Lane PMS Practice & 50.74 & 3777 & 0.013 \\
\hline
Ashgate Hospice Daycentre & 1843.03 & NULL & NULL \\
\hline
\ldots & \ldots & \ldots & \ldots \\
\hline
\end{tabular}
\end{center}
\paragraph{Conclusion}
The practice which spent most per patient was Burrswood Nursing Home, and the one which spent the least was School Lane PMS Practice. Burrswood turns up once again because it has only one patient.

\section{Question 5: What was the difference in selective serotonin reuptake inhibitor prescriptions between January and February?}
\paragraph{Interpretation}
I interpreted this question to be asking to work out the \textit{quantity} of SSRIs prescribed in January and in February and to then find the difference. The list of SSRIs I searched for, from the NHS website \cite{nhsssris}:
\begin{itemize}
\item citalopram
\item cipramil
\item dapoxetine
\item priligy
\item excitalopram
\item cipralex
\item fluoxetine
\item prozac
\item oxactin
\item fluvoxamine
\item faverin
\item paroxetine
\item seroxat
\item sertraline
\item lustral
\end{itemize}
\paragraph{Queries}
I decided to use views for this in the hope that it would speed up the search process - it worked quite well, but this was the only question I could get views to work on.
\begin{lstlisting}[language=SQL, caption=Question 5 - creating view for January SSRIs]
CREATE VIEW ssris_jan AS 
	SELECT bnf_name NAME, 
	items ITEMS, 
	period PERIOD 
	FROM prescriptions 
	WHERE period=201601 
		AND (bnf_name LIKE '%citalopram%' 
		OR bnf_name LIKE '%cipramil%' 
		OR bnf_name LIKE '%dapoxetine%' 
		OR bnf_name LIKE '%priligy%' 
		OR bnf_name LIKE '%escitalopram%' 
		OR bnf_name LIKE '%cipralex%' 
		OR bnf_name LIKE '%fluoxetine%' 
		OR bnf_name LIKE '%prozac%' 
		OR bnf_name LIKE '%oxactin%' 
		OR bnf_name LIKE '%fluvoxamine%' 
		OR bnf_name LIKE '%faverin%' 
		OR bnf_name LIKE '%paroxetin%' 
		OR bnf_name LIKE '%seroxat%' 
		OR bnf_name LIKE '%sertraline%' 
		OR bnf_name LIKE '%lustral%');
\end{lstlisting}

\begin{lstlisting}[language=SQL, caption=Question 5 - creating view for February SSRIs]
CREATE VIEW ssris_jan AS 
	SELECT bnf_name NAME, 
	items ITEMS, 
	period PERIOD 
	FROM prescriptions 
	WHERE period=201602
		AND (bnf_name LIKE '%citalopram%' 
		OR bnf_name LIKE '%cipramil%' 
		OR bnf_name LIKE '%dapoxetine%' 
		OR bnf_name LIKE '%priligy%' 
		OR bnf_name LIKE '%escitalopram%' 
		OR bnf_name LIKE '%cipralex%' 
		OR bnf_name LIKE '%fluoxetine%' 
		OR bnf_name LIKE '%prozac%' 
		OR bnf_name LIKE '%oxactin%' 
		OR bnf_name LIKE '%fluvoxamine%' 
		OR bnf_name LIKE '%faverin%' 
		OR bnf_name LIKE '%paroxetin%' 
		OR bnf_name LIKE '%seroxat%' 
		OR bnf_name LIKE '%sertraline%' 
		OR bnf_name LIKE '%lustral%');
\end{lstlisting}
\begin{lstlisting}[language=SQL, caption=Question 5 - finding SSRIs prescribed in January]
SELECT SUM(items) AS January_SSRIs FROM ssris_jan;
\end{lstlisting}
\begin{lstlisting}[language=SQL, caption=Question 5 - finding SSRIs prescribed in February]
SELECT SUM(items) AS February_SSRIs FROM ssris_feb;
\end{lstlisting}
\paragraph{Results}
\begin{center}
\begin{tabular}{ | c | }
\hline
January SSRIs \\
\hline
2742049\\
\hline
\end{tabular}
\end{center}

\begin{center}
\begin{tabular}{ | c | }
\hline
February SSRIs \\
\hline
2725157 \\
\hline
\end{tabular}
\end{center}

2742049 - 2725157 = 16892
\paragraph{Conclusion}
16,892 fewer SSRIs were prescribed in February 2016 than in January 2016.

\section{Question 6: Visualise the top 10 practices by number of metformin prescriptions throughout the entire period}
\paragraph{Interpretation}
I interpreted this question to be asking for a table showing the name, id, postcode and metformin prescribed of the top 10 practices by metformin prescribed.
\paragraph{Queries}
At first I tried using views to answer this query, but I was not able to do so without the views being corrupted and taking a huge amount of time to query, so I decided to use nested SELECT statements instead.
\begin{lstlisting}[language=SQL, caption=Question 6 query]
SELECT 
	(SELECT address.practice_name 
		FROM address 
		WHERE address.practice_id = prescriptions.practice_id GROUP BY practice_id) 
	AS practice_name, 
	SUM(items) AS metformin_prescribed 
	FROM prescriptions 
	WHERE 
		bnf_name LIKE '%metformin%' 
	GROUP BY prescriptions.practice_id 
	ORDER BY metformin_prescribed DESC 
	LIMIT 10;
\end{lstlisting}
\paragraph{Results}
\begin{center}
\begin{tabular}{ | c | c | }
\hline
Practice Name & Metformin Prescribed \\
\hline
Midlands Medical Partnership & 3192 \\
\hline
Lakeside Healthcare & 2848 \\
\hline
Spinney Hill Medical Centre & 2810 \\
\hline
The Shrewsbury Centre & 2739 \\
\hline
Beacon Medical Practice & 2652 \\
\hline
Marisco Medical Practice & 2545 \\
\hline
Vida Healthcare & 2264 \\
\hline
Harford Health Centre & 2183 \\
\hline
Portsdown Group Practice & 2108 \\
\hline
Coastal Medical Group & 2083 \\
\hline
\end{tabular}
\end{center}

\chapter{Observations on the Dataset}

\chapter{Conclusions}

\addcontentsline{toc}{chapter}{Code Listings}
\lstlistoflistings
\addcontentsline{toc}{chapter}{Bibliography}
\begin{thebibliography}{9}
	\bibitem{xampp}
		XAMPP (2017)
		Available at: \\https://www.apachefriends.org/index.html
	\bibitem{nhsprescriptiondata}
		NHS (2016)
		Available at: \\https://data.gov.uk/dataset/prescribing-by-gp-practice-presentation-level
	\bibitem{nhspatientdata}
		NHS(2016)
		Available at: \\{https://data.gov.uk/dataset/numbers\textunderscore of\textunderscore patients\textunderscore registered\textunderscore at\textunderscore a\textunderscore gp\textunderscore practice}
	\bibitem{nhsbetablockers}
		NHS (2017)
		\textit{Beta-blockers}.
		Available at: \\{http://www.nhs.uk/conditions/beta-blockers/pages/introduction.aspx}
	\bibitem{nhsssris}
		NHS (2017)
		\textit{SSRIs}.
		Available at: \\{http://www.nhs.uk/conditions/SSRIs-(selective-serotonin-reuptake-inhibitors)/Pages/Introduction.aspx}
\end{thebibliography}
\end{document}